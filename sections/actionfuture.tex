Nous implémentons actuellement les indicateurs recueillis lors des interviews. Avec les données obtenues, nous avons  une vision des sources de données d'où nous pouvons extraire les informations nécessaires au calcul des indicateurs sélectionnés par les différentes tâches du projet. 
L'implémentation se fait en plusieurs étapes. La première consiste à fournir une entrée utilisateur directe pour la saisie d'indicateurs bruts calculés manuellement. Cette initiale implémentation permet de commencer le suivi le plus tôt possible pendant la mise en place du processus et du pipeline de données. 
Parallèlement, l'utilisation de la Narration du Changement nous permettra de débuter l'étude d'impact le temps qu'un ensemble significatif de données ait pu être collecté. Cette narration permet d'interpréter les perspectives de chaque tâche sur leur environnement au niveau du projet, de l'alliance et de l'environnement de chaque institution de l'alliance.