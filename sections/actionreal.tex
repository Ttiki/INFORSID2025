Notre initiative s'aligne avec le collectif universitaire européen, UNITA - Universitas Montium, qui regroupe 12 établissements académiques dédiés à la poursuite d'objectifs communs. Cette alliance sert de terrain d'essai, mettant en avant les obstacles que doivent surmonter les universités européennes — et plus largement, les méta-organisations — pour évaluer leur impact sociétal. Notre stratégie repose sur un cadre pluridimensionnel qui aide les méta-organisations comme UNITA à mesurer leur influence grâce à des méthodologies basées sur les données et des outils analytiques. 
Notre méthode se déploie en trois étapes distinctes : \textbf{(1) Discussion} : Déterminer les besoins spécifiques de chaque membre. \textbf{(2) Évaluation} : Valider l'adéquation et la fiabilité des indicateurs clés. \textbf{(3) Construction} : Intégrer les données validées dans un Entrepôt de Données (DW). 
Au cours de cette première année, nous avons adopté une méthodologie organisée pour rassembler les indicateurs et les besoins spécifiques aux différents travaux de l'alliance UNITA. Par le biais d'entretiens collaboratifs, nous avons dressé et approuvé une liste d'indicateurs clés. Après validation, chaque indicateur est prêt à être collecté auprès des institutions membres pour initier les analyses telles que l'élaboration de rapports et la prévision de l'impact. 
