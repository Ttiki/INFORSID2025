Notre initiative s'aligne avec le collectif universitaire européen, UNITA - Universitas Montium, qui regroupe 12 établissements académiques dédiés à la poursuite d'objectifs communs. Cette alliance sert de terrain d'essai audacieux, mettant en avant les obstacles que doivent surmonter les universités européennes — et plus largement, les méta-organisations — pour évaluer leur impact sociétal. Notre stratégie repose sur un cadre pluridimensionnel qui aide les méta-organisations comme UNITA à mesurer leur influence grâce à des méthodologies basées sur les données et des outils analytiques. Nous avons développé une solution de stockage orientée impact qui intègre la collecte de preuves, l'analyse d'indicateurs, et des entretiens collectifs. Au cœur de cette méthode se trouvent les indicateurs de production (résultats immédiats, comme le nombre de projets réalisés) et les indicateurs de résultats (effets à moyen terme, tels que de meilleures opportunités de collaboration) qui jouent un rôle clé dans le suivi des progrès et le soutien des décisions stratégiques. Notre méthode se déploie en trois étapes distinctes : \textbf{ (1) Discussion} : Déterminer les besoins spécifiques de chaque membre. \textbf{(2) Évaluation} : Valider l'adéquation et la fiabilité des indicateurs clés. \textbf{(3) Construction} : Intégrer les données validées dans un Entrepôt de Données (DW). 

Au cours de cette première année, nous avons adopté une méthodologie bien organisée pour rassembler les indicateurs et les besoins spécifiques aux différents travaux de l'alliance UNITA. Par le biais d'entretiens collaboratifs, nous avons dressé et approuvé une liste d'indicateurs clés en étroite collaboration avec chaque équipe. Cette étape aboutit à travers une seconde série d'entretiens spécialement dédiée à la validation des données. Une fois les données purgées de toute incohérence et doublon, elles se trouvent intégrées dans un DW centralisé. Ce dispositif simplifie le suivi des avancées, la détection des grandes lignes et la prise de décisions stratégiques éclairées. Après validation, chaque indicateur est prêt à être collecté auprès des institutions membres pour initier les analyses telles que l'élaboration de rapports et la prévision de l'impact. 
