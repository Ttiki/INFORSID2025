La concentration sur les alliances universitaires a fortement augmenté depuis le lancement en 2019 de l'Initiative des Universités Européennes (IUE). À mesure que ces partenariats se développent, il est crucial d'évaluer leur impact sociétal et de formuler des stratégies de suivi complètes. Cette analyse explore la recherche axée sur la mesure de l'impact, la Théorie du Changement (TdC) et les systèmes de Suivi et d'Évaluation (S\&E) au sein de méta-organisations comme les alliances universitaires. En investiguant ces sujets, nous visons à établir un cadre solide pour évaluer leur impact. Dans un premier temps, nous allons éclaircir les concepts de base.

%Add a subsection on meta-organisations
\subsection{Étude d'Impact et Théorie du Changement}
%\begin{figure}
%    \centering
%    \includegraphics[width=0.75\linewidth]{images/Diagrams-IMPACT.png}
%    \caption{Chaîne d'impact simplifiée \cite{peersman_when_2016}}
%    \label{fig:simplified-impact-chain}
%\end{figure}
Le terme ‘impact’ fait souvent référence aux effets à long terme des activités sur leur environnement. Nous adoptons la définition du Comité d'Aide au Développement (CAD) de l’Organisation de Coopération et de Développement Économiques (OCDE) : "les effets à long terme positifs et négatifs, primaires et secondaires produits par une intervention de développement, directement ou indirectement, intentionnels ou non." \cite{oecd_quality_2010}. Un outil souvent utilisé dans l'étude d'impact est la chaîne d'impact simplifiée comme expliqué par \cite{peersman_when_2016}. Constituée des inputs en entrées qui seront utilisés pour différentes activités et actions. Puis, en résultat de ces actions; nous aurons des outputs, qui mènent à un plus haut niveau aux outcomes de l'action puis en fin de chaîne nous avons l'impact de nos actions. 
%La figure \ref{fig:simplified-impact-chain} présente une chaîne d'impact simplifiée reliant les ressources, activités, résultats et effets \cite{peersman_when_2016}.

La théorie du changement a évolué au fil du temps, débutant avec le "Modèle de Changement en Trois Étapes" de Kurt Lewin \cite{lewin_frontiers_1947} et se développant à travers des modèles cycliques élargis par Lippitt et al. \cite{lippitt_dynamics_1958} et Prochaska et DiClemente \cite{prochaska_stages_1983}. Ces théories, semblables à la Théorie de l’Action Raisonnée et du Comportement Planifié, intègrent des notions clés comme "le contrôle perçu", mais manquent d'une perspective sociétale. Nous examinons la théorie du changement dans le domaine de l'Alphabétisation Politique, fusionnant une approche politiquement informée et stratégique, cruciale pour planifier, suivre et évaluer les changements au sein des méta-organisations. Elle nous permet, entre autres, d'utiliser le concept de la Narration du Changement pour planifier l'étude d'impact.

\subsection{Système de Suivi et d'Évaluation automatisé orienté données}
Les systèmes de Suivi et d'Évaluation (S\&E) combinent des évaluations régulières pour des mesures quantitatives avec une surveillance continue pour identifier les anomalies, assurant des données fiables tout au long du cycle de vie d'un projet. Avec notre approche orientée données, nous nous intéressons au cadre proposé et faisons le lien avec la sous-section suivante.
Les entrepôts de données orientés vers la connaissance centralisent et exploitent des informations souvent abstraites comme l'impact sociétal. Ils fournissent une perspective unifiée au sein de l'alliance, k, facilitant des évaluations précises grâce à des indicateurs statistiques. Un entrepôt de données 