Les alliances universitaires ont attiré l'attention, notamment depuis le lancement en 2019 de l'Initiative des universités européennes (IUE). À mesure que les alliances s'étendent, il est nécessaire d'évaluer leur impact sociétal et de développer des stratégies de suivi. Cet examen décrit la recherche sur la mesure de l'impact, la Théorie du Changement (TdC) et les systèmes de Suivi et d'Évaluation (S\&E) au sein de méta-organisations comme les alliances universitaires. En explorant ces sujets, nous visons à construire un cadre pour évaluer leur impact. Nous définirons d'abord les concepts clés de notre étude.

\subsection{Impact}
L'impact désigne généralement les effets à long terme des activités sur leur environnement. Nous retenons la définition du Comité d'Aide au Développement (CAD) de l'Organisation de Coopération et de Développement Économiques (OCDE) : "effets à long terme positifs et négatifs, primaires et secondaires, produits par une intervention de développement, directement ou indirectement, intentionnels ou non" \cite{oecd_quality_2010}.

L'identification d'indicateurs pertinents reste un défi. La Figure \ref{fig:simplified-impact-chain} illustre une chaîne d'impact simplifiée reliant ressources, activités, résultats et effets \cite{peersman_when_2016}.

\begin{figure}
    \centering 
    \includegraphics[width=1\linewidth]{images/Diagrams-IMPACT.png} 
    \caption{Chaîne d'impact simplifiée\cite{peersman_when_2016}}
    \label{fig:simplified-impact-chain} 
\end{figure}

\subsection{Théorie du changement}
La théorie du changement a évolué au fil du temps, passant de la "Théorie des trois étapes du changement" de Kurt Lewin\cite{lewin_frontiers_1947} aux modèles élargis et cycliques proposés par Lippitt et al.\cite{lippitt_dynamics_1958} et Prochaska et DiClemente\cite{prochaska_stages_1983}. Ces théories, comme la Théorie de l’action raisonnée et du comportement planifié, introduisent des concepts clés tels que le "contrôle perçu", mais manquent d’une perspective sociétale. 

Nous explorons la TDC de la Littératie Politique, qui combine une approche politiquement informée et stratégique, essentielle pour la planification, le suivi et l'évaluation des changements au sein des méta-organisations.

\subsection{Système de Suivi et d’Évaluation}
Les systèmes de Suivi et d’Évaluation (S\&E) combinent une évaluation régulière pour quantifier les mesures et un suivi continu pour détecter les anomalies, garantissant ainsi des données fiables tout au long du cycle de vie d’un projet.

\subsection{Data warehouse et base de connaissances partagée}
Les entrepôts de données orientés connaissances permettent de centraliser et d’exploiter des informations souvent abstraites, comme l’impact sociétal. Ils offrent une vision commune à travers l’alliance, facilitant une évaluation précise via des indicateurs statistiques.
