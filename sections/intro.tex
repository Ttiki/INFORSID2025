La globalisation et l'internationalisation de l'éducation ont connu une accélération spectaculaire ces dernières années, en particulier au sein des établissements d'enseignement supérieur. Ce mouvement mondial s'est popularisé en Europe grâce à l'initiative Erasmus+ des universités européennes. En 2019, l'initiative des universités européennes (EUI) a été lancée avec pour mission de promouvoir, de faciliter et de diffuser la nécessité d'alliances interuniversitaires en Europe. Depuis, 64 alliances, couvrant plus de 35 pays, regroupant plus de 500 universités, ont été créées. Ces alliances entraînent des changements et des transformations importants, apportant de nouveaux besoins et défis. Pour assurer le développement et la durabilité des universités européennes à venir, il est nécessaire de surveiller et d'évaluer l'impact lié à la vision à long terme de ces alliances.

Les alliances européennes pourraient être caractérisées en suivant le modèle proposé en 2005 par G. Ahrne et N. Brunsson \cite{ahrne_organizations_2005}, en appliquant le concept de méta-organisations pour désigner le regroupement de nombreuses organisations autour d'objectifs et de projets communs. 