Les méta-organisations, définies comme des groupements d'organisations autour d’objectifs communs, émergent dans des domaines variés : académique, industriel, territorial, etc. Leur gouvernance distribuée, la diversité de leurs membres, et la multiplicité de leurs projets rendent l’évaluation de leur impact complexe. Cette étude propose un framework générique de suivi et d’évaluation de l’impact (S\&E) adapté à ces structures. Notre proposition intègre des technologies sémantiques au sein d’un data warehouse pour faciliter l’agrégation, l’interprétation et l’exploitation de données hétérogènes issues de plusieurs organisations.
Ce travail est expérimenté dans le cadre de l’alliance universitaire européenne UNITA, utilisée ici comme cas d’usage représentatif d’une méta-organisation, sans en restreindre la portée générale.