La notion de méta-organisation et l'étude de l'impact existent depuis plusieurs années, mais il manque un cadre méthodologique dédié pour évaluer leur impact, particulièrement dans le contexte des alliances universitaires européennes qui nécessitent une collaboration interinstitutionnelle. Cette recherche vise à combler ce vide en créant un framework de suivi et d'évaluation basé sur les données, spécialisé pour les méta-organisations. Elle s'articule autour de trois axes clés :
\textbf{(1) Définition et caractéristiques :} Comment définir une méta-organisation dans le cadre universitaire ? Les alliances universitaires européennes y correspondent-elles ou doivent-elles être reclassées ?
\textbf{(2) Méthodologie d'impact :} Quelles méthodes sont appropriées pour évaluer l'impact des méta-organisations ? Comment établir des indicateurs pertinents à différents niveaux (alliance vs institution) ? Les méthodes classiques telles que le modèle de la Théorie du Changement sont-elles applicables ?
\textbf{(3) Suivi technique et organisationnel :} Comment gérer des données hétérogènes de diverses institutions via un data warehouse sémantique ? Quels mécanismes d'interopérabilité et de gouvernance de données assurent une coordination efficace tout en respectant les spécificités institutionnelles ?